\section{Introduction}
In reinforcement learning (RL),
actor-critic approaches combine policy- and value-based methods~\cite{6392457};
pure versions of those methods are called actor-only and critic-only, respectively.
Essensially, we have a parameterized policy in the actor that is
updated with information from the critic.
The objective is to maximize the expected $\gamma$-discounted cumulative return,
$J(\theta) = \mathbb{E}_{\pi} [R_t] = \mathbb{E}_{\pi} [ \sum_{i \ge 0}^{\infty} \gamma^i r(s_{t+i}, a_{t+1}) ]$
with respect to the policy parameters~$\theta$.
The policy gradient is defined as
$\nabla_{\theta} J(\theta) = \mathbb{E}_{\pi} [ \sum_{i \ge 0}^{\infty} \Psi^t~\nabla log~\pi_{\theta} (a_t | s_t) ]$,
where $\Psi^t$ is typically an advantage function $A^{\pi}(s_t,a_t)$ computed by the critic.
The trend is to utilize deep neural networks as both policy representation and
(non-linear) function approximator, by which we have nonconvex optimization and
the so-called deep actor-critic RL.

\section{Issues}
Here, we review the current issues: problems and their existing solutions.
We focus on actor-critic approaches that are non-hierarchical and
consider fully-observable single task and single agent.

\subsection{Low sample efficiency (high sample complexity)}
This problem stems from the fact that vanilla actor-critic methods are on-policy only.
On-policy learning leads to good convergence to the target policy at
the cost of low sample efficiency as it trains on each trajectory only once.

One straightforward solution is to (also) train off-policy using a replay buffer.
This, however, results in bad convergence because the training data
do not come from the current policy, but from another behaviour policy.
Consequently, works in this line propose ideas that not only benefits from off-policy learning
but also (try to) maintains convergence.
This includes Reactor~\cite{Gruslys2018}, Q-prop~\cite{}, and ACER~\cite{}.

Reactor proposes prioritized sequence replay.
Q-prop utilizes Taylor expansion of the critic as a control variate.
ACER introduces truncated importance sampling with bias correction and
stochastic dueling network architectures.

Other solutions for low sample officiency do not incorporate off-policy learning
(so they stay on-policy), for example: ACKTR~\cite{}, soft-AC~\cite{}, and A3C~\cite{}.

ACKTR uses more advance optimization technique, i.e. Kronecker-factored approximate curvature (K-FAC).
Whereas, soft-AC uses maximum entropy.

\subsection{Bias and variance tradeoff}
Actor-critic methods originate from policy-based methods that
use Monte-carlo rollouts to estimate the policy gradient; hence having high variance.
The value from critic is aimed to reduce the variance and to accelerate learning;
although it introduces bias and an asymptotic dependence on the quality of the function approximation.
As a result, the intricacy between actor and critic can be encoded as the bias and variance tradeoff.

Reactor~\cite{} proposes beta-leave-one-out policy gradient estimate.
GAC~\cite{} introduces guide actors.
TD3~\cite{} favors underestimations.
Dual-AC~\cite{} suggests using dual critic.

\subsection{Reproducibility}
The reproducibility issue in actor-critic is naturally worse than in general RL~\cite{}.
Typically, there are, at least, 2 neural networks in deep actor-critic RL methods.
Although it is theoretically possible to use one network for both actor and critic,
in practice, especially for continuous action spaces, separate networks are superior~\cite{A3c}.
This implies that there are more hyperparameters to tune.
For example, DDPG~\cite{} requires 4 neural networks, in addition,
its performance is known to be sensitive to hyperparameter settings~\cite{}.

There exist several initiatives for better reproducibility.
In our opinion, the top-2 are OpenAI's baseline~\cite{} and gym~\cite{}.
The former provides high quality implementation of major actor-critic algorithms.
Whereas, the latter has a set of standardized test environments for both discrete and continuous action spaces.
We observe that the community progressively accepts these baseline and gym for benchmarking.
The progress, however, has been slowed down mainly due to technical choices of the deep-net backend,
whether tensorflow (as in the baseline) or pytorch.

\section{Discussions}
In order to foster the reproducibility, we should aim for an architecture with the following characterisics.
First, it is minimal, for example, in term of the number of neural networks.
Concretely, this means one shared neural network for both actor and critic.
Secondly, it has insensitive hyperparams so that rough tuning is already sufficient
Thirdly, it provide an easy switch among neural-network backends.

Because sample complexity remains high, it is always desirable to have parallel implementation.
Its distributive nature will reduce wall-clock time so that more experiment can be carried out.
We note that most works report statistics inferred from few number of experiments; merely 3, 5, up to 10 runs.
The parallelism should not limited to the deep network training, but to include core component of actor-critic.
One good attempt is Reactor~\cite{} that presents distributional retrace.
