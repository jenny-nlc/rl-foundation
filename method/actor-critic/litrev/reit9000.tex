\section{Introduction}
In reinforcement learning (RL),
actor-critic approaches combine policy- and value-based methods~\cite{6392457};
pure versions of those methods are called actor-only and critic-only, respectively.
Essensially, we have a parameterized policy in the actor that is
updated with information from the critic.

The objective is to maximize the expected $\gamma$-discounted cumulative return,
$J(\theta) = \mathbb{E}_{\pi} [R_t] = \mathbb{E}_{\pi} [ \sum_{t \ge 0}^{\infty} \gamma^t r(s_{t}, a_{t}) ]$
with respect to the policy parameters~$\theta$.
The policy gradient is defined as
$\nabla_{\theta} J(\theta) = \mathbb{E}_{\pi} [ \sum_{t \ge 0}^{\infty} \Psi^t~\nabla log~\pi_{\theta} (a_t | s_t) ]$,
where $\Psi^t$ is typically an advantage function $A^{\pi}(s_t,a_t)$ computed by the critic.
The trend is to utilize deep neural networks as both policy representation and
(non-linear) function approximator, by which we have nonconvex optimization and
the so-called deep actor-critic RL.

\section{Issues}
Here, we review the current issues: problems and their existing solutions.
We focus on actor-critic approaches that are non-hierarchical and
consider fully-observable single task and single agent.

\subsection{Low sample efficiency (high sample complexity)}
This problem stems from the fact that vanilla actor-critic methods are on-policy only.
On-policy learning leads to good convergence to the target policy at
the cost of low sample efficiency as it trains on each trajectory only once.\\

\noindent
\textbf{Off-policy solutions:}\\
One straightforward solution for this is to train off-policy using a replay buffer.
This, however, results in bad convergence because the training data
do not come from the current policy, but from another behaviour policy.
Consequently, works in this line propose ideas that \emph{not only} benefit from
off-policy learning \emph{but also} (try to) maintain convergence.

Reactor~\cite{Gruslys2018} exploits the temporal locality of neighboring observations for
more efficient replay prioritization.
In Reactor, critic is trained by the multi-step off-policy Retrace algorithm,
while actor is trained by beta-leave-one-out policy gradient estimate.
Q-prop~\cite{GuLilGhaTurLev17} uses the first-order Taylor expansion of the critic as a control variate,
This yields an analytical gradient term through the critic and
a Monte Carlo policy gradient term consisting of the residuals in advantage approximations.
Overall, one can view Q-prop as
using the off-policy critic to reduce variance in policy gradient, or
using on-policy Monte Carlo returns to correct for bias in the critic gradient.

SAC~\cite{abs-1801-01290} combines off-policy actor-critic training with a stochastic actor, and
further aims to maximize the entropy of this actor with an entropy maximization objective.
ACER~\cite{WangBHMMKF16} introduces 2 ways in harnessing experience replay, namely:
truncated importance sampling with bias correction and
stochastic dueling network architectures.\\

\noindent
\textbf{On-policy solutions:}\\
Other solutions for low sample officiency do not incorporate off-policy learning.
In other words, they stay on-policy.
We notice that fewer works are carried out using on-policy solutions for dealing with high sample complexity.

ACKTR~\cite{NIPS2017_7112} optimizes both the actor and the critic using
Kronecker-factored approximate curvature (K-FAC) with trust region.
It also applies a natural gradient update to the critic;
in most works, natural gradient updates are only for the actor.
A3C~\cite{pmlr-v48-mniha16} asynchronously executes multiple agents in parallel on multiple instances of the environment;
instead of using experience replay.
The parallelism decorrelates the agents’ data into a more stationary process, since
at any given time-step the parallel agents will be experiencing a variety of different states.

\subsection{Bias and variance tradeoff}
Actor-critic methods originate from policy-based methods in model-free setting that
wait until the end of an episode to estimate the policy gradient; hence having high variance.
The value from critic is aimed to reduce the variance and to accelerate learning;
although it introduces bias and an asymptotic dependence on the quality of the function approximation.
As a result, the intricacy between actor and critic can be encoded as the bias and variance tradeoff.

GAC~\cite{tangkaratt2018guide} learns a guide actor that locally maximizes the critic.
It updates the guide actor by performing second-order optimization in
the action space where the curvature matrix is based on the Hessians of the critic.
TD3~\cite{abs-1802-09477} favors underestimations.
It takes the minimum value between a pair of critics to restrict overestimation and
delays policy updates to reduce per-update error.
In Dual-AC~\cite{dai2018boosting}, the actor and dual-critic are updated cooperatively to
optimize the same objective function.
This can be viewed as a two-player game between the actor and a critic-like function.

\subsection{Low reproducibility}
The reproducibility issue in actor-critic approaches is naturally worse than in general RL~\cite{henderson2017reinforcement}.
Typically, there are, at least, 2 neural networks in deep actor-critic RL.
Although it is theoretically possible to use one network for both actor and critic,
in practice, especially for continuous action spaces, separate networks are superior~\cite{pmlr-v48-mniha16}.
This implies that there are more hyperparameters to tune.
For example, DDPG~\cite{LillicrapHPHETS15} requires 4 neural networks.
Moreover, its performance is known to be sensitive to hyperparameter settings~\cite{Duan:2016}.

There exist several initiatives for better reproducibility in terms of standardization.
In our opinion, the top-2 are OpenAI's baseline~\cite{baselines} and gym~\cite{1606.01540}.
The former provides high quality implementation of major actor-critic algorithms.
Whereas, the latter has a set of standardized test environments for both discrete and continuous action spaces.
We observe that the community progressively accepts these baseline and gym for benchmarking.
The progress, however, has been slowed down mainly due to technical choices of the deep-net backend,
whether tensorflow (as in the baseline) or pytorch.

\section{Discussions}
We believe that we should not trade convergence for sample efficiency.
In other words, the first preference is to learn on-policy with some advance optimization methods
in order to improve the sample efficiency, like ACKTR~\cite{NIPS2017_7112}.
To this end, other network architecture, including activation functions, are also worth exploring.
Additionally, we argue that variance can be reduced \emph{not only} by critic,
\emph{but also} by other variance reduction techniques from the field of monte-carlo simulation~\cite{citeulike:14544227}.

In order to foster the reproducibility, we should aim for an architecture with the following characterisics.
First, it is minimal, for example, in terms of the number of neural networks.
Concretely, this means one shared neural network for both actor and critic.
Secondly, it has insensitive hyperparams so that rough tuning is already sufficient.
Thirdly, it provides an easy switch among neural-network backends.

Because sample complexity remains high, it is always desirable to have parallel implementation.
Its distributive nature will reduce wall-clock time so that more experiments can be carried out.
We note that most works report statistics inferred merely from few number of experiments; 3, 5, up to 10 runs~\cite{henderson2017reinforcement}.
The parallelism should not only for deep network training, but also for core components of actor-critic.
One good attempt is Reactor~\cite{Gruslys2018} that presents distributional retrace.
