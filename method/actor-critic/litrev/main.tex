%%%%%%%%%%%%%%%%%%%%%%%%%%%%%%%%%%%%%%%%%%%%%%%%%%%%%%%%%%%%%%%%%%%%%%%%%%%%%%%%
%2345678901234567890123456789012345678901234567890123456789012345678901234567890
%        1         2         3         4         5         6         7         8

\documentclass[letterpaper, 10 pt, conference]{ieeeconf}  % Comment this line out if you need a4paper

%\documentclass[a4paper, 10pt, conference]{ieeeconf}      % Use this line for a4 paper

\IEEEoverridecommandlockouts                              % This command is only needed if
                                                          % you want to use the \thanks command

\overrideIEEEmargins                                      % Needed to meet printer requirements.

% See the \addtolength command later in the file to balance the column lengths
% on the last page of the document

% The following packages can be found on http:\\www.ctan.org
\usepackage{graphics} % for pdf, bitmapped graphics files
\usepackage{epsfig} % for postscript graphics files
\usepackage{mathptmx} % assumes new font selection scheme installed
\usepackage{times} % assumes new font selection scheme installed
\usepackage{amsmath} % assumes amsmath package installed
\usepackage{amssymb}  % assumes amsmath package installed

%%%%%%%%%%%%%%%%%%%%%%%%%%%%%%%%%%%%%%%%%%%%%%%%%%%%%%%%%%%%%%%%%%%%%%%%%%%%%%%%
% https://tex.stackexchange.com/questions/247104/hyperref-doesnt-link-cite-command
% https://tex.stackexchange.com/questions/50747/options-for-appearance-of-links-in-hyperref
\makeatletter
\let\NAT@parse\undefined
\makeatother
\usepackage{hyperref}
\hypersetup{
  colorlinks=true,
  linkcolor=blue,
  filecolor=magenta,
  urlcolor=blue,
  citecolor=blue
}

\usepackage[yyyymmdd,hhmmss]{datetime}

%%%%%%%%%%%%%%%%%%%%%%%%%%%%%%%%%%%%%%%%%%%%%%%%%%%%%%%%%%%%%%%%%%%%%%%%%%%%%%%%
\title{\LARGE \bf
Lit rev: actor-critic approaches in reinforcement learning
}

\author{Vektor Dewanto (s44513791)$^{*}$
\thanks{$^{*}$ The annotated bibliography is available at
\url{https://github.com/tttor/rl-foundation/tree/master/method/actor-critic}, @ver~\today~\currenttime}%
}

\begin{document}

\maketitle
\thispagestyle{empty}
\pagestyle{plain}

%%%%%%%%%%%%%%%%%%%%%%%%%%%%%%%%%%%%%%%%%%%%%%%%%%%%%%%%%%%%%%%%%%%%%%%%%%%%%%%%
\section{Introduction}
In reinforcement learning (RL),
actor-critic approaches combine policy- and value-based methods~\cite{6392457};
pure versions of those methods are called actor-only and critic-only, respectively.
Essensially, we have a parameterized policy in the actor that is
updated with information from the critic.

The objective is to maximize the expected $\gamma$-discounted cumulative return,
$J(\theta) = \mathbb{E}_{\pi} [R_t] = \mathbb{E}_{\pi} [ \sum_{i \ge 0}^{\infty} \gamma^i r(s_{t+i}, a_{t+1}) ]$
with respect to the policy parameters~$\theta$.
The policy gradient is defined as
$\nabla_{\theta} J(\theta) = \mathbb{E}_{\pi} [ \sum_{i \ge 0}^{\infty} \Psi^t~\nabla log~\pi_{\theta} (a_t | s_t) ]$,
where $\Psi^t$ is typically an advantage function $A^{\pi}(s_t,a_t)$ computed by the critic.
The trend is to utilize deep neural networks as both policy representation and
(non-linear) function approximator, by which we have nonconvex optimization and
the so-called deep actor-critic RL.

\section{Issues}
Here, we review the current issues: problems and their existing solutions.
We focus on actor-critic approaches that are non-hierarchical and
consider fully-observable single task and single agent.

\subsection{Low sample efficiency (high sample complexity)}
This problem stems from the fact that vanilla actor-critic methods are on-policy only.
On-policy learning leads to good convergence to the target policy at
the cost of low sample efficiency as it trains on each trajectory only once.\\

\noindent
\textbf{Off-policy solutions:}\\
One straightforward solution for this is to (also) train off-policy using a replay buffer.
This, however, results in bad convergence because the training data
do not come from the current policy, but from another behaviour policy.
Consequently, works in this line propose ideas that \emph{not only} benefits from
off-policy learning \emph{but also} (try to) maintains convergence,
for example, Reactor~\cite{Gruslys2018}, Q-prop~\cite{}, and ACER~\cite{}.

Reactor~\cite{Gruslys2018} exploits the temporal locality of neighboring observations for
more efficient replay prioritization.
In Reactor, critic is trained by the multi-step off-policy Retrace algorithm,
while actor is trained by beta-leave-one-out policy gradient estimate.
Q-prop~\cite{} uses the first-order Taylor expansion of the critic as a control variate,
This yields an analytical gradient term through the critic and
a Monte Carlo policy gradient term consisting of the residuals in advantage approximations.
Overall, one can view Q-prop as
using the off-policy critic to reduce variance in policy gradient, or
using on-policy Monte Carlo returns to correct for bias in the critic gradient.
ACER~\cite{} introduces 2 ways in using experience replay, namely:
truncated importance sampling with bias correction and
stochastic dueling network architectures.\\

\noindent
\textbf{On-policy solutions:}\\
Other solutions for low sample officiency do not incorporate off-policy learning
(so they stay on-policy), for example: ACKTR~\cite{}, soft-AC~\cite{}, and A3C~\cite{}.

ACKTR uses more advance optimization technique, i.e. Kronecker-factored approximate curvature (K-FAC).
Whereas, soft-AC uses maximum entropy.

\subsection{Bias and variance tradeoff}
Actor-critic methods originate from policy-based methods that
use Monte-carlo rollouts to estimate the policy gradient; hence having high variance.
The value from critic is aimed to reduce the variance and to accelerate learning;
although it introduces bias and an asymptotic dependence on the quality of the function approximation.
As a result, the intricacy between actor and critic can be encoded as the bias and variance tradeoff.

GAC~\cite{} introduces guide actors.
TD3~\cite{} favors underestimations.
Dual-AC~\cite{} suggests using dual critic.

\subsection{Reproducibility}
The reproducibility issue in actor-critic is naturally worse than in general RL~\cite{}.
Typically, there are, at least, 2 neural networks in deep actor-critic RL methods.
Although it is theoretically possible to use one network for both actor and critic,
in practice, especially for continuous action spaces, separate networks are superior~\cite{A3c}.
This implies that there are more hyperparameters to tune.
For example, DDPG~\cite{} requires 4 neural networks, in addition,
its performance is known to be sensitive to hyperparameter settings~\cite{}.

There exist several initiatives for better reproducibility.
In our opinion, the top-2 are OpenAI's baseline~\cite{} and gym~\cite{}.
The former provides high quality implementation of major actor-critic algorithms.
Whereas, the latter has a set of standardized test environments for both discrete and continuous action spaces.
We observe that the community progressively accepts these baseline and gym for benchmarking.
The progress, however, has been slowed down mainly due to technical choices of the deep-net backend,
whether tensorflow (as in the baseline) or pytorch.

\section{Discussions}
reduce variance not only by critic, but variance reduction technique from monte-carlo sim field;
should not trade convergence for sample efficiency;
should be on-policy with other optimization tehck, optiomizer, activation fn, net arch

In order to foster the reproducibility, we should aim for an architecture with the following characterisics.
First, it is minimal, for example, in term of the number of neural networks.
Concretely, this means one shared neural network for both actor and critic.
Secondly, it has insensitive hyperparams so that rough tuning is already sufficient
Thirdly, it provide an easy switch among neural-network backends.

Because sample complexity remains high, it is always desirable to have parallel implementation.
Its distributive nature will reduce wall-clock time so that more experiment can be carried out.
We note that most works report statistics inferred from few number of experiments; merely 3, 5, up to 10 runs.
The parallelism should not limited to the deep network training, but to include core component of actor-critic.
One good attempt is Reactor~\cite{} that presents distributional retrace.

% \section{Introduction}

One promising approach in RL is actor-critic.
Briefly, it takes advantage of both value- and policy-based RL.
One can view that actor-critic methods originate from policy-based RL, ie. REINFORCE.

The REINFORCE-with-baseline algorithm learns both a policy and a state-value function.
However, it does not belong to actor-critic becuase
its state-value function is used only as a baseline, not as a critic.
In other words, the value function is not used for bootstrapping
(updating the value estimate for a state from the estimated values of subsequent states),
but only as a baseline for the state whose estimate is being updated.
Such bootstrapping is useful to reduces variance and accelerates learning,
although it introduce bias and an asymptotic dependence on
the quality of the function approximation.

Figure~\ref{} shows REINFORCE-with-baseline and one-step vanilla actor-critic.
The latter is on-policy actor-critic.
TODO: iterate the diff.

Challenges include:
\begin{itemize}
\item data/sample efficiency,
\item stable, good convergence
\end{itemize}

The trend:
\begin{itemize}
\item deep neural networks as function approximators.
\item off-policy learning
\item scales to both continuous and discrete action spaces
\end{itemize}

Survey on actor-critic include:
\cite{6392457}

% \section{Problems and Existing Solutions}

\subsection{low sample efficiency}
AKA
low data efficiency
hi sample complexity

WHY:
* on-policy learning; train on each trajectory only once,
  * recall: vanilla actor-critic methods are on-policy only
  * on-policy leads to good convergence, why?

SOLUTION:
* off-policy learning (via replay buffer),
  BUT
  * may lead to bad convergence, why?
  EG:
  * reactor~\cite{Gruslys2018}: prioritized sequence replay
  * qprop:  Taylor expansion of the critic as a control variate
  * acer: 3 contrib:
    truncated importance sampling with bias correction,
    stochastic dueling network architectures, and
    a new trust region policy optimization method.

* use more advanced optimization techniques
EG
* acktr: kfac

\subsection{unguaranteed convergence}

AKA
* sensitive to hyperparameter settings

WHY
* off-policy learning

SOLUTION
* qprop: Taylor expansion of the critic as a control variate

CONCERNS
* off-policy learning
* nonconvex optim in net
* nonlinear fn approx of net

\subsection{computational efficiency}
AKA
computation speed,
time efficiency,
training time

WHY
* applying natural grad

SOLUTION
* paralel (distributive nature)
  EG:
  * reactor
    * distributional retrace

CONCERN
* gpu vs cpu computation

\subsection{tradeoff bias and variance in estimating the gradient}
AKA
* overestimation on value estimate

SOLUTION
* xxx
  EG:
  * reactor: beta-leave-one-out policy gradient estimate
* xxx2
  EG:
  * GAC: guide actor
* favors underestimations
  EG:
  * TD3
* dual critic

\subsection{ease of implementation, deployment}
AKA
reproducibility


% \section{Evaluation}
% use reported values,
% eval metric

% Identify components

% * psi: the advantage that is put in gradient

% * optimization method: adam, rmsprop, sgd, kfac
%   * order: first, second

% * constraint in opt: KL-div

% * OFF - ON POLICY
%   off-policy learning: for sample efficiency at the cost of learning stability

% * ACTOR AND CRITIC NETS: shared, separated

% To ensure exploration in the policy
% * to use entropy regularization to ensure exploration in the policy,
% which is a common practice in policy gradient (Williams & Peng, 1991; Mnih et al., 2016).
% * to use KL-divergence as a constraint on how much deviation is permitted from a prior policy
% (Bagnell & Schneider, 2003; Peters et al., 2010; Schulman et al., 2015; Fox et al., 2015).

% \section{Taxonomy}

% \subsection{On- vs -off policy learning}
% On-policy:
% A3C~\cite{Mnih2016},
% ACKTR,
% natural AC,

% Off-policy:
% Reactor~\cite{Gruslys2018},
% SoftAC~\cite{Haarnoja2017},
% ACER~\cite{Wang2016},
% using experience replay buffer,

% \subsection{stoc vs det policy}

% \subsection{Discrete action spaces: Atari tasks}
% Atari game~\footnote{\url{https://gym.openai.com/envs/\#atari}}.

% Properties:
% \begin{itemize}
% \item discrete action spaces
% \item 57 games
% \end{itemize}

% Reactor: LSTM net

% \subsection{Continuous action space: Mujoco tasks}
% Mujoco tasks~\footnote{\url{https://gym.openai.com/envs/\#mujoco}}

% Properties:
% \begin{itemize}
% \item continuous action spaces
% \item 6 task: cartpole (1D), reacher (3D), cheetah (6D), fish (5D), walker (6D) and humanoid (21D)
% \end{itemize}

% \begin{table}[]
% \centering
% \caption{My caption}
% \label{my-label}

% \begin{tabular}{|l|l|l|l|}
% \hline
% Methods & Setup & Atari Tasks & Mujoco Tasks \\ \hline
% ACKTR   & s1    & a1          & m1           \\ \hline
% ACER    & s2    & a2          & m2           \\ \hline
% \end{tabular}

% \end{table}

% \section{Discussions}
% net arch
% which optimizer? Adam, KFAC, SGL, RMSProp
% activation fn

% ranks
% ACKTR
% A2C

% Variance Reduction Techniques,
% * ch 10 @Sheldon M. Ross-Simulation, Fifth Edition-Academic Press (2012)
% * https://people.smp.uq.edu.au/DirkKroese/montecarlohandbook/
% * Monte Carlo theory, methods and examples, Art B. Owen, 2013

% not the same set of task,
% eg reacher is missing

% reduce variance will reduce sample complexity

We believe that we should not trade convergence for sample efficiency.
In other words, the first preference is to learn on-policy with some advance optimization methods
in order to improve the sample efficiency, like ACKTR~\cite{NIPS2017_7112}.
To this end, other network architectures, including activation functions, learning rates, are also worth exploring.
Additionally, we argue that variance can be reduced \emph{not only} by critic,
\emph{but also} by specific variance reduction techniques from the field of monte-carlo simulation~\cite{citeulike:14544227}.

In order to foster the reproducibility, we should aim for an architecture with the following characterisics.
First, it is minimal, for example, in terms of the number of neural networks.
Concretely, this means one shared neural network for both actor and critic.
Secondly, it has insensitive hyperparameters so that rough tuning is already sufficient.
Thirdly, it provides easy switching among neural-network backends.

Because sample complexity remains high, it is always desirable to have parallel implementation.
Its distributive nature will reduce wall-clock time so that more experiments can be carried out.
We note that most works report statistics inferred merely from few number of experiments; 3, 5, up to 10 runs~\cite{henderson2017reinforcement}.
The parallelism should not only for deep network training, but also for core components of actor-critic.
One good attempt is Reactor~\cite{Gruslys2018} that presents distributional retrace.


%%%%%%%%%%%%%%%%%%%%%%%%%%%%%%%%%%%%%%%%%%%%%%%%%%%%%%%%%%%%%%%%%%%%%%%%%%%%%%%%
\addtolength{\textheight}{-12cm}   % This command serves to balance the column lengths
                                  % on the last page of the document manually. It shortens
                                  % the textheight of the last page by a suitable amount.
                                  % This command does not take effect until the next page
                                  % so it should come on the page before the last. Make
                                  % sure that you do not shorten the textheight too much.

%%%%%%%%%%%%%%%%%%%%%%%%%%%%%%%%%%%%%%%%%%%%%%%%%%%%%%%%%%%%%%%%%%%%%%%%%%%%%%%%
% \section*{APPENDIX}
% Appendixes should appear before the acknowledgment.

%%%%%%%%%%%%%%%%%%%%%%%%%%%%%%%%%%%%%%%%%%%%%%%%%%%%%%%%%%%%%%%%%%%%%%%%%%%%%%%%
% \section*{ACKNOWLEDGMENT}
% TODO

%%%%%%%%%%%%%%%%%%%%%%%%%%%%%%%%%%%%%%%%%%%%%%%%%%%%%%%%%%%%%%%%%%%%%%%%%%%%%%%%
\bibliographystyle{apalike}
\bibliography{main}

\end{document}
