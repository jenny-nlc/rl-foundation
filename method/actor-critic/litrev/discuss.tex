\section{Discussions}
% net arch
% which optimizer? Adam, KFAC, SGL, RMSProp
% activation fn

% ranks
% ACKTR
% A2C

% Variance Reduction Techniques,
% * ch 10 @Sheldon M. Ross-Simulation, Fifth Edition-Academic Press (2012)
% * https://people.smp.uq.edu.au/DirkKroese/montecarlohandbook/
% * Monte Carlo theory, methods and examples, Art B. Owen, 2013

% not the same set of task,
% eg reacher is missing

% reduce variance will reduce sample complexity

% on- / off- policy / merging both
% model, experience replay for sample efficiency

We believe that we should not trade convergence for sample efficiency.
In other words, the first preference is to learn on-policy with some advance optimization methods
in order to improve the sample efficiency, like ACKTR~\cite{NIPS2017_7112}.
To this end, other network architectures, including activation functions, learning rates, are also worth exploring.
Additionally, we argue that variance can be reduced \emph{not only} by critic,
\emph{but also} by specific variance reduction techniques from the field of monte-carlo simulation~\cite{citeulike:14544227}.

In order to foster the reproducibility, we should aim for an architecture with the following characterisics.
First, it is minimal, for example, in terms of the number of neural networks.
Concretely, this means one shared neural network for both actor and critic.
Secondly, it has insensitive hyperparameters so that rough tuning is already sufficient.
Thirdly, it provides easy switching among neural-network backends.

Because sample complexity remains high, it is always desirable to have parallel implementation.
Its distributive nature will reduce wall-clock time so that more experiments can be carried out.
We note that most works report statistics inferred merely from few number of experiments; 3, 5, up to 10 runs~\cite{henderson2017reinforcement}.
The parallelism should not only for deep network training, but also for core components of actor-critic.
One good attempt is Reactor~\cite{Gruslys2018} that presents distributional retrace.
